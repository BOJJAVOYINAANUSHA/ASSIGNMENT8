\documentclass[journal,12pt,twocolumn]{IEEEtran}
%
\usepackage{setspace}
\usepackage{gensymb}
\singlespacing
\usepackage[cmex10]{amsmath}
\usepackage{siunitx}
\usepackage{amsthm}

\usepackage{mathrsfs}

\usepackage{txfonts}
\usepackage{stfloats}

\usepackage{steinmetz}
\usepackage{cite}
\usepackage{cases}
\usepackage{subfig}
\usepackage{longtable}
\usepackage{multirow}
\usepackage{enumitem}
\usepackage{mathtools}
\usepackage{tikz}
\usepackage{circuitikz}
\usepackage{verbatim}
\usepackage{tfrupee}
\usepackage[breaklinks=true]{hyperref}
\usepackage{tkz-euclide} % loads  TikZ and tkz-base
\usetikzlibrary{calc,math}
\usetikzlibrary{fadings}
\usepackage{listings}
    \usepackage{color}                                            %%
    \usepackage{array}                                            %%
    \usepackage{longtable}                                        %%
    \usepackage{calc}                                             %%
    \usepackage{multirow}                                         %%
    \usepackage{hhline}                                           %%
    \usepackage{ifthen}                                           %%
  %optionally (for landscape tables embedded in another document): %%
    \usepackage{lscape}     
\usepackage{multicol}
\usepackage{chngcntr}
\DeclareMathOperator*{\Res}{Res}

\renewcommand\thesection{\arabic{section}}
\renewcommand\thesubsection{\thesection.\arabic{subsection}}
\renewcommand\thesubsubsection{\thesubsection.\arabic{subsubsection}}

\renewcommand\thesectiondis{\arabic{section}}
\renewcommand\thesubsectiondis{\thesectiondis.\arabic{subsection}}
\renewcommand\thesubsubsectiondis{\thesubsectiondis.\arabic{subsubsection}}

\hyphenation{op-tical net-works semi-conduc-tor}
\def\inputGnumericTable{}                                 %%

\lstset{
%language=C,
frame=single, 
breaklines=true,
columns=fullflexible
}
\begin{document}
%


\newtheorem{theorem}{Theorem}[section]
\newtheorem{problem}{Problem}
\newtheorem{proposition}{Proposition}[section]
\newtheorem{lemma}{Lemma}[section]
\newtheorem{corollary}[theorem]{Corollary}
\newtheorem{example}{Example}[section]
\newtheorem{definition}[problem]{Definition}
\newcommand{\BEQA}{\begin{eqnarray}}
\newcommand{\EEQA}{\end{eqnarray}}
\newcommand{\define}{\stackrel{\triangle}{=}}
\bibliographystyle{IEEEtran}
\providecommand{\mbf}{\mathbf}
\providecommand{\pr}[1]{\ensuremath{\Pr\left(#1\right)}}
\providecommand{\qfunc}[1]{\ensuremath{Q\left(#1\right)}}
\providecommand{\sbrak}[1]{\ensuremath{{}\left[#1\right]}}
\providecommand{\lsbrak}[1]{\ensuremath{{}\left[#1\right.}}
\providecommand{\rsbrak}[1]{\ensuremath{{}\left.#1\right]}}
\providecommand{\brak}[1]{\ensuremath{\left(#1\right)}}
\providecommand{\lbrak}[1]{\ensuremath{\left(#1\right.}}
\providecommand{\rbrak}[1]{\ensuremath{\left.#1\right)}}
\providecommand{\cbrak}[1]{\ensuremath{\left\{#1\right\}}}
\providecommand{\lcbrak}[1]{\ensuremath{\left\{#1\right.}}
\providecommand{\rcbrak}[1]{\ensuremath{\left.#1\right\}}}
\theoremstyle{remark}
\newtheorem{rem}{Remark}
\newcommand{\sgn}{\mathop{\mathrm{sgn}}}
\providecommand{\abs}[1]{\left\vert#1\right\vert}
\providecommand{\abs}[1]{\lvert#1\rvert} 
\providecommand{\res}[1]{\Res\displaylimits_{#1}} 
\providecommand{\norm}[1]{\left\lVert#1\right\rVert}
%\providecommand{\norm}[1]{\lVert#1\rVert}
\providecommand{\mtx}[1]{\mathbf{#1}}
\providecommand{\mean}[1]{E\left[ #1 \right]}
\providecommand{\fourier}{\overset{\mathcal{F}}{ \rightleftharpoons}}
%\providecommand{\hilbert}{\overset{\mathcal{H}}{ \rightleftharpoons}}
\providecommand{\system}{\overset{\mathcal{H}}{ \longleftrightarrow}}
	%\newcommand{\solution}[2]{\textbf{Solution:}{#1}}
\newcommand{\solution}{\noindent \textbf{Solution: }}
\newcommand{\cosec}{\,\text{cosec}\,}
\providecommand{\dec}[2]{\ensuremath{\overset{#1}{\underset{#2}{\gtrless}}}}
\newcommand{\myvec}[1]{\ensuremath{\begin{pmatrix}#1\end{pmatrix}}}
\newcommand{\mydet}[1]{\ensuremath{\begin{vmatrix}#1\end{vmatrix}}}
\numberwithin{equation}{subsection}
\makeatletter
\@addtoreset{figure}{problem}
\makeatother
\let\StandardTheFigure\thefigure
\let\vec\mathbf
\renewcommand{\thefigure}{\theproblem}
\def\putbox#1#2#3{\makebox[0in][l]{\makebox[#1][l]{}\raisebox{\baselineskip}[0in][0in]{\raisebox{#2}[0in][0in]{#3}}}}
     \def\rightbox#1{\makebox[0in][r]{#1}}
     \def\centbox#1{\makebox[0in]{#1}}
     \def\topbox#1{\raisebox{-\baselineskip}[0in][0in]{#1}}
     \def\midbox#1{\raisebox{-0.5\baselineskip}[0in][0in]{#1}}
\vspace{3cm}
\title{ASSIGNMENT-8}
\author{B.ANUSHA}
\maketitle
\newpage
\bigskip
\renewcommand{\thefigure}{\theenumi}
\renewcommand{\thetable}{\theenumi}
%
\section{QUESTION No-2.26(Matrices)}
Using elementary transformations, find the inverse of each of the matrices:
\begin{enumerate}
\item \myvec {2 & 1 \\ 1 & 1}
\end{enumerate}
\section{Solution}
\begin{enumerate}
\item Given that
\begin{align}
\vec{A} = \myvec{2 & 1 \\ 1 & 1}
\end{align}
The augmented matrix $ [A | I]$ is as given below:- 
\begin{align}
\myvec{2 & 1 & \vrule & 1 & 0 \\ 1 & 1 & \vrule & 0 & 1}
\end{align}
We apply the elementary row operations on $ [A | I]$ as follows :-
\begin{align}
[A | I] = \myvec{2 & 1 & \vrule & 1 & 0 \\ 1 & 1 & \vrule & 0 & 1}
\\
\xleftrightarrow{R_1\leftarrow R_1-R_2}   
\myvec{1 & 0 & \vrule & 1 & -1 \\ 1 & 1 & \vrule & 0 & 1}
\\
\xleftrightarrow{R_2\leftarrow R_2-R_1}
\myvec{1 & 0 & \vrule & 1 & -1 \\ 0 & 1 & \vrule & -1 & 2}
\end{align}
By performing elementary transformations on augmented matrix$ [A | I]$ , we obtained the augmented matrix in the form $ [I | A]$. 
Hence we can conclude that the matrix A is invertible and inverse of the matrix is:-
\begin{align}
\therefore\vec{A^{-1}}=\myvec {1 & -1 \\  -1 & 2} 
\end{align}
\item QR decomposition of  \myvec{2 & 1 \\ 1 & 1}
\\
Let $\alpha$ and $\beta$ be the column vectors of given matrix $\vec{A}$,
\begin{align}
\alpha=\myvec{2 \\ 1},
\beta=\myvec{1 \\ 1} \label{1.0}
\end{align}
QR decomposition of matrix form is:
\begin{align}
\myvec{\alpha & \beta}=\myvec{\vec{u_1} & \vec{u_2}}\myvec{k_1 & r_1 \\ 0 & k_2} \label{1.1}
\end{align}
Finding values of the above equation, we get:
\begin{align}
\implies k_1&=\norm{\alpha}=\norm{\myvec{2 \\ 1}}
\\
\therefore k_1&=\sqrt{5} \label{1.2}
\\
\implies \vec{u_1}&=\frac{\alpha}{k_1}=\frac{1}{\sqrt{5}}\myvec{2 \\ 1}
\\
\therefore \vec{u_1}&=\myvec{\frac{2}{\sqrt{5}} \\ \frac{1}{\sqrt{5}}} \label{1.3}
\\
\implies r_1&=\frac{\vec{u_1}^{T}{\beta}}{\norm{\vec{u_1}}^2}=\frac{\frac{1}{\sqrt{5}}\myvec{2 & 1}\myvec{1 \\ 1}}{1}
\\
\therefore r_1&=\frac{3}{\sqrt{5}} \label{1.4}
\\
\implies \vec{u_2}&=\frac{\beta-r_1{\vec{u_1}}}{\norm{\beta-r_1{\vec{u_1}}}}=\frac{\myvec{1\\1}-\frac{3}{\sqrt{5}}\myvec{\frac{2}{\sqrt{5}} \\ \frac{1}{\sqrt{5}}}}{\norm{\myvec{1\\1}-\frac{3}{\sqrt{5}}\myvec{\frac{2}{\sqrt{5}} \\ \frac{1}{\sqrt{5}}}}}
\\
\therefore \vec{u_2}&=\myvec{\frac{-1}{\sqrt{5}} \\ \frac{2}{\sqrt{5}}} \label{1.5}
\\
\implies k_2&=\vec{u_2}^{T}{\beta}=\myvec{\frac{-1}{\sqrt{5}} & \frac{2}{\sqrt{5}}}\myvec{1 \\ 1}
\\
\therefore k_2&=\frac{1}{\sqrt{5}} \label{1.6}
\end{align}
From equations \eqref{1.2}, \eqref{1.3}, \eqref{1.4}, \eqref{1.5}, \eqref{1.6} and using \eqref{1.0} the obtained $\vec{Q}\vec{R}$ decomposition is 
\begin{align}
\myvec{2 & 1 \\ 1 & 1}=\myvec{\frac{2}{\sqrt{5}} & \frac{-1}{\sqrt{5}} \\ \frac{1}{\sqrt{5}} & \frac{2}{\sqrt{5}}}\myvec{\sqrt{5} & \frac{3}{\sqrt{5}} \\ 0 & \frac{1}{\sqrt{5}}}
\end{align}
\end{enumerate}
\end{document}